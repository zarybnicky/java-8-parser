% arara: pdflatex: { options: "--jobname great-name" }
\documentclass[12pt,a4paper]{report}
\usepackage[utf8]{inputenc}
\usepackage[margin=1.8cm]{geometry}
\usepackage{amsmath}
\usepackage[czech, slovak]{babel}
\usepackage{amsfonts}
\usepackage{amssymb}
\usepackage{graphicx}
\usepackage{lmodern}

% TODO
% -

% FIXME img link, compiling with make [Table of Content]

\begin{document}



\begin{titlepage}

\fontfamily{qhv}\selectfont
\addtolength{\voffset}{-3cm}

\noindent\hspace{-60.0pt}\includegraphics[width=0.6\textwidth]{./img/VUT_symbol_barevne_CMYK_CZ}\\
\Large\textbf{VYSOKÉ UČENÍ TECHNICKÉ V BRNĚ}\\
\small\textbf{BRNO UNIVERSITY OF TECHNOLOGY}\\\vfill\hspace{-0.7cm}
\Large\textbf{FAKULTA INFORMAČNÍCH TECHNOLOGIÍ}\\
\small\textbf{FACULTY OF INFORMATION TECHNOLOGY}\\\vfill\hspace{-0.7cm}
\Large\textbf{IFJ - DOKUMENTACE PROJEKTU}\\
\small\textbf{IFJ - PROJECT DOCUMENTATION}\\\vfill\hspace{-0.7cm}
\Large\textbf{SEMESTRÁLNÍ PROJEKT}\\
\small\textbf{TERM PROJECT}\\\vfill\hspace{-0.7cm}
\Large\textbf{AUTORI PRÁCE \hfill \normalsize{Marek Tamaškovič, Martin Vaško,}}\\
\small\textbf{AUTHORS} \hfill \normalsize\textbf{Michal Vaško, Jirka Záleský, Jakub Zárybnický}
\vfill\hspace{-0.7cm}
\large\textbf{BRNO 2017}

\newpage
\fontfamily{\familydefault}\selectfont%
\end{titlepage}

\newgeometry{margin=2.5cm}

\begin{abstract}
Nas abstrakt\\
http://www.fit.vutbr.cz/info/szz/bib\_citace.html\\
http://www.fit.vutbr.cz/info/szz/psani\_textu.php\\
http://citace.info/norma1/webova-stranka/
\end{abstract}

\tableofcontents

\chapter{Implementačné detaily}

\section{Lexikálna analýza}
asdf
\section{Syntaktická analýza}
asdf
\section{Sémantická analýza}
asdf
\section{Interpret}
Interpret má za úlohu vykonať to, čo sa nachádza v zdrojovom kóde interpretovaného programu. Náš interpret interpretuje abstraktný syntaktický strom, ktorý vytvorila syntaktická analýza. Ten lineárne prechádza a vyhodnocuje výrazy pokým nenastane volanie funkcie. V tom momente si vyhľadá v tabuľke symbolov abstraktný syntaktický strom danej funkcie, vytvorí lokálnu tabuľku symbolov, ktorú bude používať volaná funkcia, vloží do nej argumenty funkcie a začne vykonávať telo funkcie. Pri ukončovaní funkcie interpret vloží návratovú hodnotu z funkcie na zásobník a ukončí interpretáciu funkcie. Následne si interpret danú hodnotu vyberie zo zásobníka a použije ju v interpretácii pôvodného abstraktného syntaktického stromu. Vstavané funkcie sú riešené obdobne. Taktiež si pri interpretácii interpret kontroluje behové chyby ako napríklad delenie nulou alebo práca s neinicializovanými premennými. Ak taká situácia nastane interpret sa ukončuje s chybovou hláškou a príslušným návratovým kódom pre danú chybu.


\chapter{Príloha}

príloha

\end{document}
